\documentclass[a4paper,11pt]{article}
\usepackage[left=1.2cm, right=1.5cm, top=1.6cm, bottom=1.3cm]{geometry}
\usepackage{xeCJK}
\usepackage{indentfirst}
\usepackage{tabularx}
\usepackage{float}
\usepackage{graphicx}
\usepackage{amsmath}
\usepackage{listings}
\usepackage{verbatim}
\usepackage{fancyhdr}
% \usepackage[compact]{titlesec}
\usepackage[usenames,dvipsnames]{xcolor}

\setCJKmainfont{NotoSansCJKtc-Thin}
\setmonofont{Consolas}

\definecolor{CodeGreen}{rgb}{0,0.6,0}
\definecolor{CodeGray}{rgb}{0.5,0.5,0.5}
\definecolor{CodeMauve}{rgb}{0.58,0,0.82}
\lstset{
    basicstyle = \ttfamily\footnotesize, 
    breakatwhitespace = false,
    breaklines = true,         
    commentstyle = \color{CodeGreen}\bfseries,
    extendedchars = false,
    keepspaces=true,
    keywordstyle=\color{blue}\bfseries, % keyword style
    language = C++,                     % the language of code
    otherkeywords={string},
    numbers=left,
    numbersep=5pt,
    numberstyle=\tiny\color{CodeGray},
    rulecolor=\color{black},
    showspaces=false,
    showstringspaces=false,
    showtabs=false,
    stepnumber=1,       
    stringstyle=\color{CodeMauve},        % string literal style
    tabsize=2,
}
% from https://blog.csdn.net/RobertChenGuangzhi/article/details/45126785

\title{Machine Learning 2020 - Homework 10 Report}
\author{學號:b08902100, 系級:資工一, 姓名:江昱勳}
\date{}

\begin{document}
\pagestyle{fancy}
\fancyhead[L]{Machine Learning 2020 - Homework 10}
\fancyhead[R]{Author: b08902100 江昱勳}

\maketitle

% \verbatiminput{HW2_S.txt}
% \lstinputlisting{HW2.cpp}

\begin{enumerate}
    \item 任取一個baseline model (sample code裡定義的 fcn,cnn,vae) 與你在kaggle leaderboard上表現最好的單純autoencoder架構的model(如果表現最好的model就是sample code裡定義的model的話就再任選一個,e.g.  如果cnn最好那就再選fcn),對各自重建的testing data的image中選出與原圖mse最大的兩張加上最小的兩張並畫出來。(假設有五張圖,每張圖經由autoencoder A重建的圖片與原圖的MSE分別為 [25.4, 33.6, 15, 39, 54.8],則MSE最大的兩張是圖4、5而最小的是圖1、3)。須同時附上原圖與經autoencoder重建的圖片。(圖片總數:(原圖+重建)*(兩顆model)*(mse最大兩張+mse最小兩張) = 16張)
    
    \begin{figure}[H]
        \centering
        \includegraphics[width=.9\textwidth]{baseline.pdf}
        \caption{cnn 的 baseline model}
        \label{}
    \end{figure}

    \begin{figure}[H]
        \centering
        \includegraphics[width=.9\textwidth]{best.pdf}
        \caption{最好的autoencoder}
        \label{}
    \end{figure}

    上面的圖片前兩者皆是最差的兩個,後兩者皆是最好的兩個,而上方是原本的圖片,下面是autoencoder的輸出

    \item 嘗試把 sample code中的K-means 與 PCA 分別做在 autoencoder 的 encoder output 上,並回報兩者的auc score以及本來model的auc。autoencoder不限。不論分數與本來的model相比有上升還是下降,請同學簡述原因。

    在不使用PCA與KNN的情況下的score為0.62262,若使用PCA降維成16維則score為0.66144,使用K-Means分成6群的話score為0.69602。可以發現將PCA應用再autoencoder的encoder output上可以有一定幅度的提升,可以猜測是因為anomaly與正常的東西的組成不太一樣,所以使用PCA投影回去時會有明顯的不同,另外從K-Means的結果我們更可以知道,這種不同更可能是某種距離上的不同,所以直接使用K-Means來找出比較遠的點的時候會有更好的效果。

    \item 如hw9,使用PCA或T-sne將testing data投影在2維平面上,並將testing data經第1題的兩顆model的encoder降維後的output投影在2維平面上,觀察經encoder降維後是否分成兩群的情況更明顯。(因未給定testing label,所以點不須著色)

    \begin{figure}[H]
        \centering
        \includegraphics[width=.9\textwidth]{origin_pca.pdf}
        \caption{PCA on original image}
    \end{figure}

    \begin{figure}[H]
        \centering
        \includegraphics[width=.9\textwidth]{baseline_pca.pdf}
        \caption{PCA on baseline model}
    \end{figure}

    \begin{figure}[H]
        \centering
        \includegraphics[width=.9\textwidth]{best_pca.pdf}
        \caption{PCA on best model}
    \end{figure}

    可以發現越好的model會在降為後的圖形上有越集中於某一塊的取向。

    \item 說明為何使用auc score來衡量而非binary classification常用的f1 score。如果使用f1 score會有什麼不便之處?
    
    因為助教有提到這次的anomaly其實不多,而若使用f1 score的話,很有可能直接輸出皆不是anomaly會有很好的效果,但是我們希望的是能偵測出anomaly,也就是我們更偏向「寧可錯殺而不要放過一個」的模式,因此採用auc score才比較容易達到如此的目標。
\end{enumerate}

\end{document}